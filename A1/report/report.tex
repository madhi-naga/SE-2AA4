\documentclass[12pt]{article}

\usepackage{graphicx}
\usepackage{paralist}
\usepackage{listings}
\usepackage{booktabs}
\usepackage{hyperref}
\hypersetup{
    colorlinks=true,
    linkcolor=blue,
    filecolor=magenta,      
    urlcolor=cyan,
}

\oddsidemargin 0mm
\evensidemargin 0mm
\textwidth 160mm
\textheight 200mm

\pagestyle {plain}
\pagenumbering{arabic}

\newcounter{stepnum}

\title{Assignment 1 Solution}
\author{Madhi Nagarajan, nagarajm}
\date{\today}

\begin {document}

\maketitle

This purpose of this assignment is based on creating ADTs for a DateT and GPosT data-type. The DateT data-type stores a date and completes date-related calculations. The GPosT data-type stores a geographical coordinate and completes geographical calculations. Testing of these ADTs were also done to verify our personal code and our partner's code.

\section{Testing of the Original Program}

My approach to testing my program was to utilize "assert" with unittest to provide whether there were any failures or not. When testing the move and distance function, I used \url{https://www.movable-type.co.uk/scripts/latlong.html} as a basis for my test case answers. For the arrival\_date function, I first calculated the distance. I then calculate the number of days, using the distance calculated and speed (in km/days). For the final date result, used \url{https://www.timeanddate.com/date/dateadd.html}.
\newline
When testing my code, I noticed that when testing functions in pos\_adt that returned decimal values, these test cases would fail unless if both numbers were identical. Since this is unreasonable and inconvenient for testing, I used "self.assertAlmostEqual" and utilized a range parameter in which both values can differ at max between each other (eg. within 0.1 of an expected value).

\section{Results of Testing Partner's Code}

When testing Almen's code, all test functions, but "test\_arrival\_date", passed. In this test function, the test cases failed by one day. This was due to the arrival\_date of my partner's pos\_adt, as she used the ceil() function. However in my code, I used the round() function. While I believe that both ways of rounding could be valid, there was still some bias in my test cases. The Assignment Specification did not specify the rounding of days, thus I made the assumption that day value should just be rounded normally.

\section{Critique of Given Design Specification}

I liked Almen's implementation, as she used the datetime package. This made the date\_adt code simpler, as datetime took care of the date/calendar implementation. One feedback of Almen's design is to have some variable names shorter, as it would it make it easier for her when writing out code. 

\section{Answers to Questions}

\begin{enumerate}[(a)]

\item 
The state variables used for the DateT ADT were day, month, and year. The state variables used for the GPosT ADT were latitude and longitude.

\item 
DateT is immutable. This is because its methods are encapsulated. There is no way its methods can unintentionally be modified by another class. GPosT is mutable because of the move() function. When move is called, the GPosT object passed through is directly modifed.

\item 
Pytest offers a more detailed explanation of where a failure occurs for test cases. The test script could use just "assert" rather than "self.assert". While I did not use pytest, I utilized unittest (a demo of pytest) which can be used right out of the box in python. I found unittest also uses the "assert" syntax, however it does not offer an as detailed explanation for failures.

\item 
Some past examples of Software Engineering failures are NASA's Spirit Rover (shutdown because it ran out of flash memory) and the 2003 Northeast Blackout (due to a race condition).
\newline
Software quality and high cost is still a major challenge because software quality directly correlates with cost. To improve the quality of software, there is generally higher cost associated with that improvement because it's very difficult to make software correct, reliable and robust. Oftentimes, companies usually undervalue the cost and effort for a software project and as a result, it could actually lead to unwanted outcomes and costs as seen in the examples above. I believe the best way to address the challenge of software quality while reducing costs is to follow and maintain a design process best suited for a specific project. This ensures that a team is organized and efficient, while also mainitaining software quality while keeping the cost a minimal. 

\item
The rational design process is a waterfall model of how software should be designed and implemented. While the process is a rational one, it not so reasonable in many software environments. This is because there are often revisions, changes, and improvements to not only the documentation, but even the software \& code itself. 
\newline
The advantages of following documentation in a rational design process is that it gives a organized procedure such that a whole team can all understand and follow. It makes design reviews easier, increases efficiency, and the overall progress of a software project.  

\item
Correctness is related to how a software meets its requirements and specification. It is generally difficult to acheive. Robustness is how well it behaves when put under unanticipated situations, and is usually accomplished by satisfying unstated requirements. Reliability measures how reliable the software does what it is intended to do. It is more of a relative quality.

\item
Seperation of concerns is when we wish to isolate different concerns such that they won't be able to directly interact with each other. Modularity is when a system is divided into smallar subsystems called modules. Modularity utlizes seperation of concerns, as each module is considered seperate and unique. When the modules are put together, they are considered a different system.


\end{enumerate}

\newpage

\lstset{language=Python, basicstyle=\tiny, breaklines=true, showspaces=false,
  showstringspaces=false, breakatwhitespace=true}
%\lstset{language=C,linewidth=.94\textwidth,xleftmargin=1.1cm}

\def\thesection{\Alph{section}}

\section{Code for date\_adt.py}

\noindent \lstinputlisting{../src/date_adt.py}

\newpage

\section{Code for pos\_adt.py}

\noindent \lstinputlisting{../src/pos_adt.py}

\newpage

\section{Code for test\_driver.py}

\noindent \lstinputlisting{../src/test_driver.py}

\newpage

\section{Code for Partner's date\_adt.py}

\noindent \lstinputlisting{../partner/date_adt.py}

\newpage

\section{Code for Partner's pos\_adt.py}

\noindent \lstinputlisting{../partner/pos_adt.py}


\end {document}